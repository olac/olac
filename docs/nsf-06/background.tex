\section{Background and Results from Prior Support}

%%%%%%%% Scale...

Today, language technology and the linguistic sciences are confronted
with a vast array of \emph{language resources}, richly structured,
large and diverse.  Texts, recordings, dictionaries, annotations,
software, protocols, data models, file formats, newsgroups and web
indexes are just some of these resources.  The resources are growing
in size, in number, in diversity.  Multiple \emph{communities} depend
on language resources, including linguists, engineers, teachers and
actual speakers.  Many individuals and institutions serve these
communities, including archivists, software developers, and
publishers.
The number of individuals having an immediate interest in language
resources numbers in the tens of thousands.

%%%%%%%% The opportunity...

Today, we have unprecedented opportunities to \emph{connect} these
communities to the language resources they need.  First, inexpensive
mass storage technology permits large resources to be stored in
digital form, while the Extended Markup Language (XML) and Unicode
provide flexible ways to represent structured data and ensure its
long-term survival.  Second, digital publication -- both on and off
the world wide web -- is the most practical and efficient means of
sharing language resources.  Finally, the Open Language Archives
Community (OLAC) provides a standard set of language resource
descriptors (OLAC Metadata), and a portal that permits users to query
dozens of language archives using a single search.  OLAC has the
bottom-up, distributed character of the web, while simultaneously
having the efficient, structured nature of a centralized database.
This combination is well-suited to the language resource community,
where the available data is growing rapidly and where a large
user-base is consistent in how it describes its resource needs.

%%%%%%%% What OLAC has done...

OLAC was founded in 2000 with the goals of (i) developing consensus on
best current practice for the digital archiving of language resources,
and (ii) developing a network of interoperating repositories and
services for housing and accessing such resources.  For the first
goal, substantial progress on identifying best practices has now been
achieved under the NSF EMELD project; consideration of best practices
is now incorporated into the funding of language documentation
projects (DEL, \citet{BirdSimons03language}).  For the second goal,
35 repositories holding some 30,000 metadata records are
participating in OLAC; linguists can search all repositories
simultaneously, and the interest level now stands
at 100,000 queries per month.  Participating archives include:
Alaska Native Language Center Archive,
Archive of the Indigenous Languages of Latin America,
UC Berkeley Audio Archive of Linguistic Fieldwork,
Oxford Text Archive, and
SIL Language and Culture Archives.

%%%%%%%% Shortcomings...

Despite these early successes, it is still true today that ``new
digital language resources of all kinds -- lexicons, interlinear
texts, grammars, language maps, field notes, recordings -- are
difficult to reuse and less portable than the conventional printed
resources they replace'' \citep{BirdSimons03language}.  The proliferation of
language ``archiving projects'' has only made this problem more
acute: endangered languages are captured in ephemeral file formats,
and the so-called ``archive'' goes away at the end of the project.
Mirroring the twin goals of OLAC, there are just two things a linguist
must do to ensure that digital data endure: ``(1) the materials must be
put into an enduring file format, and (2) the materials must be
deposited with an archive that will make a practice of migrating them
to new storage media as needed'' (Simons 2006).  This amounts to an
injunction to individuals to \emph{use} portable formats and credible
archives.

The proposed project will create new incentives that are designed to
encourage linguists to adopt these new practices.  The project has
three objectives:

\begin{description}
  \item[Goal 1: Access to Language Archive Catalogs:]
    Improving the quality and quantity of OLAC archives.

  \item[Goal 2: Access to Language Documentation on the Web:]
    Extending coverage and access beyond the existing community of
    participants by automated cross-walking and mining.

  \item[Goal 3: Access to Digital Content:]
    Curating new digital language resources through
    language-aware institutional repositories.
\end{description}

\textbf{Prior Support:} The proposed project represents an outgrowth of
prior NSF support to the investigators in the following projects:
\textit{Multidimensional Exploration of Linguistic Databases}
  (Award \#9983258, 02/01/2000--01/31/2003);
\textit{TalkBank: A Multimedia Database of Communicative Interactions}
  (Award \#9978056, 10/01/1999--09/30/2005)
\textit{E-MELD: Electronic Metastructure for Endangered Languages Data}
  (Award \#0094934, 07/01/2001--06/30/2006);
\textit{The Rosetta Project- ALL Language Archive}
  (Award \#0333727, 11/01/2003--02/28/2006);
\textit{Querying Linguistic Databases}
  (Award \#0317826, 08/01/2003--07/31/2007).
Key results from these projects were:
(i) data models for linguistic annotations of text and speech
\citep{BirdLiberman01,MaedaBird00,GraffBird00,CottonBird02,CieriBird01,ATLAS00,BirdHarrington01}
(ii) open source linguistic annotation software
\citep{Bird01acl,MaedaBird02,BirdMaeda02,MaLee02};
(iii) OLAC, a community-wide resource discovery framework
\citep{BirdSimons00,BirdSimons00survey,BirdSimons01,BirdSimons02workshop,Simons02query,SimonsBird03lht,BirdSimons03chum,Simons03display,SimonsBird03llc,BirdSimons04metadata};
(iv) recommendations for digital storage of language data
\citep{BirdSimons03language};
(v) three CD-ROM publications of linguistic field data:
\citep{BirdBell01,Bird03paradigms,Bird03ngomba};
and
(vi) models and tools for querying linguistic databases
\citep{BirdBuneman01,BirdBunemanTan00,LaiBird04,Bird05planx,Bird06icde}
\todo{Add more citations for other PIs to the above}

% Next sub-sections lay the groundwork for our three goals,
% probably 1 page on each

\subsection{Access to Language Archive Catalogs}

%%%%%%%% The status quo...

Members of the language resources community have a common need for
discovering language resources with high precision and recall.  To
support this, the community needs to agree on
some specialized vocabularies in the metadata for
describing resources.  Over the past five years, OLAC
participants have adopted five such vocabularies:
\textit{subject language},
  for identifying precisely which which language(s) a resource is ``about'';
\textit{linguistic type},
  for classifying the structure of a resource as primary text,
  lexicon, or language description;
\textit{linguistic field},
  for specifying relevant subfields of linguistics;
\textit{discourse type},
  for indicating the linguistic genre of the material;
  and
\textit{role},
  for documenting the parts played by specific individuals and institutions
  in creating a resource.
These vocabularies are represented in the OLAC Metadata format,
which is an extension of Dublin Core Metadata \citep{BirdSimons04metadata}---the dominant metadata standard in the
digiral library and world wide web communities.
Participating language archives publish their catalogs in an XML
format, and these records are ``harvested'' twice a day by OLAC services
using the Open Archives Initiative (OAI) Protocol for Metadata Harvesting
\citep{SimonsBird03lht}---another standard of the digital library community.

%%%%%%%% How OLAC has been innovative...

OLAC has been a developer or early adopter of several other key
digital library technologies in service to the language resources
community:
(i) a simplified method for archives to publish their metadata
  (``static repositories''), lowering the barrier to entry, and
  now adopted by the OAI as a service to the wider
  digital archives community;
(ii) a Google-like search interface for searching participating
  archives with specialised support for language identification
  \citep{HughesKamat05,Hughes06lrec};
(iii) a report card system which publishes a metadata quality score
  for each participating archive to serve as a form of
  peer review and an incentive for archives to improve the quality of
  their metadata;
(iv) a metadata usage report showing how individual metadata
  elements and values have been put to use by participating archives;
(v) an ``OAI crosswalk'' permitting users of digital library services
  outside the language resources community to access all OLAC metadata;
  and
(vi) a web crawler gateway permitting users of search engines such
  as Google to access all OLAC metadata.

Beyond these technological achievements, OLAC has succeeded in
establishing a \emph{community} that operates by means of an open
process \citep{OLAC-Process}.  In a 2004 publication of the
Digital Library Federation, OLAC was singled out for this
facet of its work:

\begin{quote} \small
  OLAC is exemplary in several ways: the technical and social
  infrastructure that it has developed to support its community of
  contributors, based on shared principles and standards; the
  resources that it provides at its Web site about its purpose, scope,
  history, tools, news and events; and the efforts of its two leaders
  ... to articulate the challenges, analyze the options, and recommend
  possible solutions to their community of contributors in order to
  improve OLAC \citep{Brogan04}.
\end{quote}

%%%%%%%% Three shortcomings...

Despite these successes, OLAC has three significant shortcomings.
%% Shortcoming 1: quality
First, the quality of metadata in the majority of participating archives remains
low.  According to the OLAC report card system, the average 
metadata-quality score for an OLAC record is 6.2 out of 10. However,
this score is
inflated by four of the larger archives that are already
following best practice; the average score for the remaining 31 archives
is only 3.5 out of 10.  Another way in which quality of current metadata suffers is that it is not being kept up-to-date; most archives have not
updated their metadata to reflect new acquisitions since joining OLAC.
%% Shortcoming 2: quantity
Second, many language archives are not yet participating in OLAC.  For
example, the American Philosophical Assocation (Philadelphia) and the
National Anthropological Archives (Washington DC) are major national
centers having substantial collections of early language documentation
for native American languages.  The Oriental Institute (Chicago) and
the School of Oriental and African Studies (London) have vast
collections covering Asian and African languages.  Similarly, many
smaller archives are yet to participate in OLAC, e.g.\ the
\todo{[EXAMPLE]}, and the
Memorial University of Newfoundland Folklore and Language Archive (St
John's).  Many linguistics departments have archives holding language
documentation collected over many decades by their staff, e.g. UCLA,
UCSB, U Chicago, \todo{[MORE]}.
%% Shortcoming 3: OAIS
Third, many of the participating ``archives'' are more accurately described
as digitization projects.  Once project funding and institutional
backing cease, the materials may disappear.  Thus, these
initiatives are not yet following best practices in digital archiving,
e.g.\ as defined by the Open Archival Information Systems (OAIS)
reference model \citep{OAIS02} which has attained the status of an ISO
standard (ISO 14721:2003) and is accepted as best practice within the
digital library community.

%%%%%%%% Our vision for what needs to be done, long term...

In order to address these three shortcomings, we believe that the
language archives community needs to develop in three ways:

\begin{enumerate}\setlength{\itemsep}{0pt}
\item all OLAC repositories should have up-to-date catalogs
      that contain metadata conforming to best practice;
\item all major language archives should be participating in OLAC; and
\item all OLAC archives should conform to current best practices
      in the treatment of their holdings.
\end{enumerate}

\subsection{Access to Language Resources on the Web}

%%%%%%%% The status quo...

Language resources of all kinds have been posted on the web
and can be discovered with a standard search engine:
interlinear texts, dictionaries, linguistic descriptions, and
linguistic research papers.  Even ordinary web pages may count
as primary language documentation when they are in any of the
thousands of ``low-density'' languages, languages with a small
presence on the web.  Another source of language resources is traditional
libraries with their collections of printed dictionaries, grammars,
and linguistic monographs.  Library automation has resulted in
standards like MARC for digital cataloging and the Z39.50 protocol for 
searching digital catalogues. Today such catalogues typically have a web
interface, though their entries are usually not indexed by search
engines.  WorldCat is a single service that provides a web-based search
engine over the world's major libraries. A more recent trend in the library automation arena has been
the development of e-print repositories, often in connection with an academic institution.  These use simpler standards, Dublin Core for
metadata ctaloging and the Open Archives Initiative protocol for metdata
harvesting.  OAIster is a single service that provides a web-based
search engine over all known e-print repositories.

%%%%%%%% Technology innovations... [GFS: the topic of the follinwg is problems, not innovations]

The task of finding language resources on the web is beset with
problems.  The most obvious one is \emph{scale}: that is, searching for
low-density language resources among the billions of pages indexed
by Google is like searching for a needle in a haystack.  One project is
addressing this problem by probing web search engines with linguistic
query terms and lodging the results with the Internet Archive
\citep{BaldwinBird06}. In another case, the ODIN project has mined
the web for interlinear text examples and archived the found objects
\citep{Langendoen02,Lewis03}.  A second problem is \emph{volatility}:
for instance, a four-year longitudinal study has shown the half-life
of a web page to be approximately two years [ref to Koehler].
% GFS: Here's the reference
% http://portal.acm.org/citation.cfm?id=506072.506080&coll=ACM&dl=ACM&CFID=77507462&CFTOKEN=24608581
% Web page change and persistence---a four-year longitudinal study
% Wallace Koehler   	 
% Journal of the American Society for Information Science and Technology archive
% Volume 53 ,  Issue 2  (January 2002) Pages: 162 - 171  
The (planned) LangGator project will avoid this problem by working
with the ``historic web,'' housed at the Internet Archive.  Yet
another problem is that of the
\emph{hidden web}: namely, the content that lies behind search
interfaces and is thus opaque to conventional search engines.
Within the library automation community, JZKit is an open-source project
that addresses this by building an aggregated index of library
catalogues that are accessible thorugh about a dozen search interfaces
(including the Z39.50 and SRW/SRU standards).
% GFS: http://developer.k-int.com/jzkit2/
% Note that the same Ian Ibbotson is the contact:
% http://www.loc.gov/z3950/agency/resources/software.html

%%%%%%%% Three shortcomings...

In order for the language resources community to realize the promise of universal access to relevant resources, three major
problems still need to be addressed.
%% Shortcoming 1: low precision and recall with Google
First, due to the huge scale of the search space and the absence
of precise indexing vocabularies,
users of web search engines such as Google typically experience
low precision and recall when searching for language resources.
Searches for scarce resources are often swamped with irrelevant
results (low precision).  For example searches for ``Santa Cruz'' will
not yield results for the East Papuan language of this name, spoken in
the Solomons.
%%
Furthermore, many resources are just not returned at all because
search terms do not match the synonymous terms used in the desired
documents (low recall).  For example, searcing for ``Dschang lexicon''
will not return the Dschang lexicon developed by one of the PIs
\citep{BirdTadadjeu97}; it was published as ``Lexique Y\'emba,'' since
the language of wider communication is French, and the language
autonym for \textit{Dschang} is \textit{Y\'emba}.
%%
Such materials are straightforward to find, however, using
linguistically-aware web mining programs that probe
Google's index by querying for all synonyms and translations of
linguistic terminology and by searching for pages that contain IPA characters.
A harder problem is to reliably identify the subject language and the
linguistic type of the found resource \citep{HughesBaldwin06lrec}.

%% Shortcoming 2: hidden web
A second problem is that the libary automation solutions are part of
the hidden web and remain hidden to the language resources community
at large.
Users searching for language
resources also need to visit other services like WorldCat and OAIster;
it would be better for the language-resource content
of these services to be fully integrated with OLAC.
For example, OAI e-print repositories include many articles
relevant to languages and linguistics (see
\url{http://romeo.eprints.org/search.php?t=Linguistics}),
results which are not returned in OLAC searches.
Similarly, searching the world's libraries could be done with a
Google-mediated search of the WorldCat site, but this would not
permit systematic aggregation and integration, and would still suffer
from the problems of low recall and precision.
%% Shortcoming 3: finding OLAC
A third shortcoming is that users who try to find language resources
using any of these non-OLAC services are unlikely to discover that
OLAC can provide additional value, such as richer metadata and more
focussed result sets.

%%%%%%%% Our vision for what needs to be done, long term...

In order to address these three shortcomings, we believe that the
language archives community needs to develop in three ways:

\begin{enumerate}\setlength{\itemsep}{0pt}
\item low density language materials identified in allied projects
      should be reliably categorized for OLAC vocabularies;
\item OLAC should index all OAI and library holdings relevant to language
      documentation by crosswalking OAI-DC and MARC records and
      adding further OLAC metadata to support precise searching;
\item OLAC should export all records as static HTML pages, loaded with
      synonyms and containing embedded queries, so that
      users searching for language resources using a web search
      quickly find OLAC records and remain on the OLAC site for
      more precise searching.
\end{enumerate}

\subsection{Access to Digital Content}

\todo{[Upbeat paragraph]}

%%%%%%%% The status quo...

Most searches for archived language resources do not lead to an online version
of the resource.  In many cases, the resource is not digital, so
that it is necessary to visit an archive to access the physical artifact.
In other cases, the resource is digital but the archive has never
implemented a system for automated curation and it is necessary to
contact an archivist or an author to get a copy.  Even worse, there is a large quantity of language
documentation material in the possession of the linguists who
originally collected it that has never been archived or catalogued.
As long as they remain unarchived, they are in imminent danger of
being lost to posterity (Simons 2006). 
As long as they remain uncatalogued, searches for language resources
will never discover these resources.

% GFS: Not used
% Many sponsored language documentation projects
% create resources that only reside in the investigator's computer.
% There is some consensus about what comprises the inventory of basic
% documentation for a language (e.g. EMELD and Rosetta index pages for a
% language), but this should be broadened.

%%%%%%%% DSpace repository as solution

What the language resource community needs to solve these problems
is for language archives to have a digital archiving system that
automates the basic functions of curating a collection.  The Open Archival Information Systems (OAIS) reference model \citep{OAIS02} is an ISO standard (ISO 14721:2003) that specifies best practices for digital archiving.  It defines standards for the six basic functions that 
any archive must perform: ingest, data management, archival
storage, access, preservation planning, and administration. and is accepted as best practice within the digital library community.

In recent years, DSpace has emerged as the leading implementation of
a system that supports the OAIS reference model.  It also supports
the OAI protocol for metadata harvesting so that materials deposited
in a DSpace repository automatically become part of the virtual
digital library as indexed by OAIster and other OAI service providers.
DSpace was developed 
jointly in 2002 by the MIT Libraries and Hewlett-Packard Company to be
``MIT's online institutional repository---built to save, share, and
search MIT's digital research materials'' (\url{http://libraries.mit.edu/dspace-mit/}).
From the outset
DSpace has been distributed as an open source software product at SourceForge where it is supported by a thriving developer community
that ranks in the 98th percentile for level of activity when 
compared to all other projects (\url{http://sourceforge.net/projects/dspace/}).
It is now in use at more than one hundred universities as the institutional repository where faculty 
members log in to deposit the fruits of their research
(\url{http://wiki.dspace.org/DspaceInstances}).
The same software automatically supports search and access by the public,
including the ability to limit access where public access is not 
appropriate.  Given the large uptake by
major academic institutions along with the active open source mode of
development and dissemination, DSpace seems destined to be the leading
digital repository system for years to come. Furthermore, the future of
stored materials is ensured beyond the life of DSpace since the system
supports portable export formats for the archived information.

%%%%%%%% Three shortcomings...

Most language archives have not yet implemented automated ingest and
access for digital materials.  A system like DSpace would provide 
an off-the-shelf solution.  
However, before DSpace can meet the needs of digital language archives,
three major problems will need to be addressed.
% 1. precise language identification is missing
The first problem is the need to support precise language identification.
Language identification is the single most important thing that
characterises a linguistic resource; this is especially critical for
endangered languages---each relevant resource is highly valuable but
without precise language identification it becomes another needle in
the haystack. DSpace supports the ISO 639-2 standard for coding the
language that a resource is in \citep{ISO639-2}. Two changes are needed:
the ability to identify the language a resource is \emph{about}, and
the ability to use the ISO 639-3 standard.  ISO 639-2 provides
three-letter codes for fewer than 400 individual languages and covers all
remaining languages with 68 collective codes like [aus] ``Australian 
languages'' or [sai] ``South American Indian (other)''.  This clearly
is not acceptable for indexing low-density language resources.
By contrast, the newly developed ISO 639-3 (which is being put to
the vote for final adoption this year) provides over 7,000 additional
three-letter codes for uniquely identifying all known human languages,
past and present \citep{ISO639-3}.  ISO 639-3 has already been adopted by OLAC as its
standard for language identification.
Adding support for ISO 639-3 to DSpace would not only meet the
needs of language resource creators, but would also have the
effect of getting precise language identification into the 
digital archiving mainstream, so that cataloguers in general could
give precise identification that would already be in place
when we crosswalk from DC to OLAC.

% management of codespace over time deferred for section on method

% 2. DSpace is only for documents
A second problem is that the default configuration of DSpace
is for depositing conventional documents, such as books and articles.
However, language documentation involves other formats like
XML files, audio files, image files, and video files. Furthermore,
language resources need a richer set of metadata using the OLAC 
metadata standard with its specialized vocabularies for indexing.

% 3. fieldwork mode requires offline solution
A third problem to be solved is that linguists working
on language documentation in the field typically do not have internet access from
the fieldwork site. Thus they are unable to submit their
results to an institutional repository from the point of creation.
However, it is important to capture all the needed metadata at the time
of  creation.  Given fieldwork's ever-present risk of accident (such as
canoe tipping) or calamity (such as theft) or natural disaster (such
as tsunami) or operator error (such as inadvertent deletion), it is
also important to to be able to upload
collected materials to the institutional repository for safekeeping
as soon as it is possible to reach a location with internet access.
Therefore, the field linguist requires a solution in which archival
submissions can be created and fully documented while offline and then
uploaded later to the institutional repository.

%%%%%%%% Our vision for what needs to be done, long term...

In order to address these three shortcomings, we believe that the
language archives community needs to develop in three ways:

\begin{enumerate}\setlength{\itemsep}{0pt}
\item users of popular institutional repository solutions should be able
  to specify language identification with precision;

\item users of popular institutional repository solutions should be
    able to install an OLAC module so that they can integrate OLAC
    metadata creation into the work-flows and life-cycles of their
    digital objects;

\item field linguists should be able to participate in all the
    above while they are in the field
\end{enumerate}


