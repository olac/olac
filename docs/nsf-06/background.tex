\section{Background and Results from Prior Support}

Today, language technology and the linguistic sciences are confronted
with a vast array of \emph{language resources}, richly structured,
large and diverse.  Texts, recordings, dictionaries, annotations,
software, protocols, data models, file formats, newsgroups and web
indexes are just some of these resources.  The resources are growing
in size, in number, in diversity.  Multiple \emph{communities} depend
on language resources, including linguists, engineers, teachers and
actual speakers.  Many individuals and institutions serve these
communities, including archivists, software developers, and
publishers.  These communities are large.  For example, the
LinguistList network counts over 20,000 members.  The Linguistic Data
Consortium produces linguistic databases for some 1,000 organizations.
The number of individuals having an immediate interest in language
resources numbers in the tens of thousands.

Today, we have unprecedented opportunities to \emph{connect} these
communities to the language resources they need.  First, inexpensive
mass storage technology permits large resources to be stored in
digital form, while the Extended Markup Language (XML) and Unicode
provide flexible ways to represent structured data and ensure its
long-term survival.  Second, digital publication -- both on and off
the world wide web -- is the most practical and efficient means of
sharing language resources.  Finally, the Open Language Archives
Community (OLAC) provides a standard set of language resource
descriptors (OLAC Metadata), and a portal that permits users to query
dozens of language archives using a single search.  OLAC has the
bottom-up, distributed character of the web, while simultaneously
having the efficient, structured nature of a centralized database.
This combination is well-suited to the language resource community,
where the available data is growing rapidly and where a large
user-base is consistent in how it describes its resource needs.

OLAC was founded in 2000 with the goals of (i) developing consensus on
best current practice for the digital archiving of language resources,
and (ii) developing a network of interoperating repositories and
services for housing and accessing such resources.  For the first
goal, substantial progress on identifying best practices has now been
achieved under the NSF EMELD project; consideration of best practices
is now incorporated into the funding of language documentation
projects (DEL, \citet{BirdSimons03language}).  For the second goal,
over 30 repositories holding over 30,000 metadata records are
participating in OLAC; linguists can search all repositories
simultaneously using OLAC services, and the interest level now stands
at 100,000 queries per month.

Despite these early successes, it is still true today that ``new
digital language resources of all kinds -- lexicons, interlinear
texts, grammars, language maps, field notes, recordings -- are
difficult to reuse and less portable than the conventional printed
resources they replace'' \citep{BirdSimons03language}.  The proliferation of
language ``archiving projects'' have only made this problem more
acute: endangered languages are captured in ephemeral file formats,
and the so-called ``archive'' goes away at the end of the project.

Mirroring the twin goals of OLAC, there are just two things a linguist
must do to ensure that digital data endure: ``(1) the materials must be
put into an enduring file format, and (2) the materials must be
deposited with an archive that will make a practice of migrating them
to new storage media as needed'' (Simons 2006).  This amounts to an
injunction to individuals to *use* portable formats and credible
archives.  However, this change in practice will not happen unless
there are substantial, immediate benefits for the working linguist.

\vspace{1in}

\todo{Big picture: problem statement, vision for what needs to be done
  long-term to address the problem, approx 1 page; focus on access and
  the three different kinds of access reflected in our goals}

\todo{Stress signifance of our Language paper for scoping out the
  full range of best practice issues}

\todo{Define language resources; mention some communities that create
  and use them.}

\vspace{1in}

\textbf{Prior Support:} The current project represents an outgrowth of
prior NSF support to the investigators in the following projects:
\textit{Multidimensional Exploration of Linguistic Databases}
  (Award \#9983258, 02/01/2000--01/31/2003);
\textit{TalkBank: A Multimedia Database of Communicative Interactions}
  (Award \#9978056, 10/01/1999--09/30/2005)
\textit{E-MELD: Electronic Metastructure for Endangered Languages Data}
  (Award \#0094934, 07/01/2001--06/30/2006);
\textit{The Rosetta Project- ALL Language Archive}
  (Award \#0333727, 11/01/2003--02/28/2006);
\textit{Querying Linguistic Databases}
  (Award \#0317826, 08/01/2003--07/31/2007).
Key results from these projects were:
(i) data models for linguistic annotations of text and speech
\citep{BirdLiberman01,MaedaBird00,GraffBird00,CottonBird02,CieriBird01,ATLAS00,BirdHarrington01}
(ii) open source linguistic annotation software
\citep{Bird01acl,MaedaBird02,BirdMaeda02,MaLee02};
(iii) OLAC, a community-wide resource discovery framework
\citep{BirdSimons00,BirdSimons00survey,BirdSimons01,BirdSimons02workshop,Simons02query,SimonsBird03lht,BirdSimons03chum,Simons03display,SimonsBird03llc,BirdSimons04metadata};
(iv) recommendations for digital storage of language data
\citep{BirdSimons03language};
(v) three CD-ROM publications of linguistic field data:
\citep{BirdBell01,Bird03paradigms,Bird03ngomba};
and
(vi) models and tools for querying linguistic databases
\citep{BirdBuneman01,BirdBunemanTan00,LaiBird04,Bird05planx,Bird06icde}


\todo{Add more citations for other PIs to the above}

\todo{Next sub-sections lay the groundwork for our three goals,
  probably 1 page on each}

\subsection{Access to Language Archive Catalogs}


What OLAC does (just the facts):

- OLAC metadata \citep{BirdSimons04metadata}

- harvesting with OAI-PMH \citep{SimonsBird03lht}

- static repositories

- federated search \citep{HughesKamat05}
  - popularity \citep{Hughes06lrec}

- report card system
  - public metrics as a form of peer review

- metadata usage report, shows what metadata elements and values are in use

- OAI crosswalk (permits access by rest of digital libraries
    community; typically people searching for e-prints)

- DP9 gateway (permits search engines to index content)

- established a community

\vspace{1in}

Successes (evaluation of the above):

- quality boosted by report card system (evidence?)

- quantity boosted by static repositories, lowering the barrier to entry
  (an OLAC innovation adopted and implemented by the OAI community)

\vspace{1in}

Shortcomings:

- quality (still low); precision/recall issues (any findings from
    Baden's study?)

- quantity of archives
  - major archives not participating: APA, NAA, SOAS
  - smaller archives not participating: AILLA
  - why not?
  - tailing off of new additions

- archival credibility:
    many so-called archives are not credible archives at all,
    archiving best practices change over time

\vspace{1in}

Our vision for the future, for what the community needs:

1) all OLAC repositories should have up-to-date catalogs
   that contain metadata conforming to best practice

2) all major language archives should be in OLAC

3) all OLAC archives should conform to current best practices
    in the treatment of their holdings

\vspace{1in}

\subsection{Access to Language Documentation on the Web}

Automatic language identification
\citep{HughesBaldwin06lrec}

Finding language resources on the web
\citep{BaldwinBird06}

\vspace{1in}

\subsection{Access to Digital Content}

\vspace{1in}

