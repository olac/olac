\section{Background and Results from Prior Support}

%%%%%%%% Scale...

Today, language technology and the linguistic sciences are confronted
with a vast array of \emph{language resources}, richly structured,
large and diverse.  Texts, recordings, dictionaries, annotations,
software, protocols, data models, file formats, newsgroups and web
indexes are just some of these resources.  The resources are growing
in size, in number, in diversity.  Multiple \emph{communities} depend
on language resources, including linguists, engineers, teachers and
actual speakers.  Many individuals and institutions serve these
communities, including archivists, software developers, and
publishers.  These communities are large.  For example, the
LinguistList network counts over 20,000 members.  The Linguistic Data
Consortium produces linguistic databases for some 1,000 organizations.
The number of individuals having an immediate interest in language
resources numbers in the tens of thousands.

%%%%%%%% The opportunity...

Today, we have unprecedented opportunities to \emph{connect} these
communities to the language resources they need.  First, inexpensive
mass storage technology permits large resources to be stored in
digital form, while the Extended Markup Language (XML) and Unicode
provide flexible ways to represent structured data and ensure its
long-term survival.  Second, digital publication -- both on and off
the world wide web -- is the most practical and efficient means of
sharing language resources.  Finally, the Open Language Archives
Community (OLAC) provides a standard set of language resource
descriptors (OLAC Metadata), and a portal that permits users to query
dozens of language archives using a single search.  OLAC has the
bottom-up, distributed character of the web, while simultaneously
having the efficient, structured nature of a centralized database.
This combination is well-suited to the language resource community,
where the available data is growing rapidly and where a large
user-base is consistent in how it describes its resource needs.

%%%%%%%% What OLAC has done...

OLAC was founded in 2000 with the goals of (i) developing consensus on
best current practice for the digital archiving of language resources,
and (ii) developing a network of interoperating repositories and
services for housing and accessing such resources.  For the first
goal, substantial progress on identifying best practices has now been
achieved under the NSF EMELD project; consideration of best practices
is now incorporated into the funding of language documentation
projects (DEL, \citet{BirdSimons03language}).  For the second goal,
35 repositories holding some 30,000 metadata records are
participating in OLAC; linguists can search all repositories
simultaneously using OLAC services, and the interest level now stands
at 100,000 queries per month.

%%%%%%%% Shortcomings...

Despite these early successes, it is still true today that ``new
digital language resources of all kinds -- lexicons, interlinear
texts, grammars, language maps, field notes, recordings -- are
difficult to reuse and less portable than the conventional printed
resources they replace'' \citep{BirdSimons03language}.  The proliferation of
language ``archiving projects'' have only made this problem more
acute: endangered languages are captured in ephemeral file formats,
and the so-called ``archive'' goes away at the end of the project.
Mirroring the twin goals of OLAC, there are just two things a linguist
must do to ensure that digital data endure: ``(1) the materials must be
put into an enduring file format, and (2) the materials must be
deposited with an archive that will make a practice of migrating them
to new storage media as needed'' (Simons 2006).  This amounts to an
injunction to individuals to \emph{use} portable formats and credible
archives.  However, this change in practice will not happen unless
there are substantial, immediate benefits for the working linguist.

%%%%%%%% What we want to do about it...

\vspace{1in}

\todo{Big picture: problem statement, vision for what needs to be done
  long-term to address the problem, approx 1 page; focus on access and
  the three different kinds of access reflected in our goals}

\todo{Stress signifance of our Language paper for scoping out the
  full range of best practice issues}

\todo{Define language resources; mention some communities that create
  and use them.}

\vspace{1in}

\textbf{Prior Support:} The current project represents an outgrowth of
prior NSF support to the investigators in the following projects:
\textit{Multidimensional Exploration of Linguistic Databases}
  (Award \#9983258, 02/01/2000--01/31/2003);
\textit{TalkBank: A Multimedia Database of Communicative Interactions}
  (Award \#9978056, 10/01/1999--09/30/2005)
\textit{E-MELD: Electronic Metastructure for Endangered Languages Data}
  (Award \#0094934, 07/01/2001--06/30/2006);
\textit{The Rosetta Project- ALL Language Archive}
  (Award \#0333727, 11/01/2003--02/28/2006);
\textit{Querying Linguistic Databases}
  (Award \#0317826, 08/01/2003--07/31/2007).
Key results from these projects were:
(i) data models for linguistic annotations of text and speech
\citep{BirdLiberman01,MaedaBird00,GraffBird00,CottonBird02,CieriBird01,ATLAS00,BirdHarrington01}
(ii) open source linguistic annotation software
\citep{Bird01acl,MaedaBird02,BirdMaeda02,MaLee02};
(iii) OLAC, a community-wide resource discovery framework
\citep{BirdSimons00,BirdSimons00survey,BirdSimons01,BirdSimons02workshop,Simons02query,SimonsBird03lht,BirdSimons03chum,Simons03display,SimonsBird03llc,BirdSimons04metadata};
(iv) recommendations for digital storage of language data
\citep{BirdSimons03language};
(v) three CD-ROM publications of linguistic field data:
\citep{BirdBell01,Bird03paradigms,Bird03ngomba};
and
(vi) models and tools for querying linguistic databases
\citep{BirdBuneman01,BirdBunemanTan00,LaiBird04,Bird05planx,Bird06icde}


\todo{Add more citations for other PIs to the above}

\todo{Next sub-sections lay the groundwork for our three goals,
  probably 1 page on each}

\subsection{Access to Language Archive Catalogs}

%%%%%%%% What OLAC currently provides...

Members of the language resources community share a common goal of
wanting to search for language resourceswith high recall
and precision. To support this, the metadata records of the
community need to agree on the use of specialized vocabularies.
Over the past five years, OLAC participants have adopted five
such vocabularies:
\textit{subject language},
  for identifying precisely which which language(s) it is about;
\textit{linguistic type},
  for classifying the structure of a resource as primary text,
  lexicon, or language description;
\textit{linguistic field},
  for specifying relevant subfields of linguistics;
\textit{discourse type},
  for indicating the linguistic genre of the material;
\textit{role},
  for documenting the role that specific individuals and institutions
  have played in creating a given resource.
These vocabularies are represented in the OLAC Metadata format,
an extension of Dublin Core Metadata \citep{BirdSimons04metadata}.
Participating language archives publish their catalogs in an XML
format, and these are ``harvested'' twice a day by OLAC services,
using the Open Archives Initiative (OAI) Protocol for Metadata Harvesting
\citep{SimonsBird03lht}.

%%%%%%%% How OLAC has been innovative...

OLAC has been a developer or early adopter of several other key
digital library technologies in service to the language resources
community:
(i) a simplified method for archives to publish their metadata
  (``static repositories''), now adopted by the OAI as a service
  to the wider digital archives community, lowering the barrier to entry;
(ii) a Google-like search interface for searching participating
  archives, with specialised support for language identification
  \citep{HughesKamat05,Hughes06lrec};
(iii) a report card system, which publishes a metadata quality score
  for each participating archive, to serve as a form of
  peer review, and an incentive for archives to improve the quality of
  their metadata;
(iv) a metadata usage report, showing how individual metadata
  elements and values have been put to use by participating archives;
(v) an ``OAI crosswalk'', permitting users of digital library services
  outside the language resources community to access all OLAC metadata;
(vi) a web crawler gateway, permitting users of search engines such
  as Google to access all OLAC metadata.

Beyond these technological achievements, OLAC has succeeded in
establishing a \emph{community}.  In a 2004 publication of the
Digital Library Federation, OLAC was singled out for this
facet of its work:

\begin{quote} \small
  OLAC is exemplary in several ways: the technical and social
  infrastructure that it has developed to support its community of
  contributors, based on shared principles and standards; the
  resources that it provides at its Web site about its purpose, scope,
  history, tools, news and events; and the efforts of its two leaders
  ... to articulate the challenges, analyze the options, and recommend
  possible solutions to their community of contributors in order to
  improve OLAC \citep{Brogan04}.
\end{quote}

\vspace{1in}

%%%%%%%% Three shortcomings...

Despite these successes, OLAC still has significant shortcomings in
three areas.
%% Shortcoming 1
First, the quality of metadata is still low.  The average score for an
OLAC record is 6/10, a score which is inflated by a small number of
large archives that are already following best practice.  The majority
of archives deliver records with a lower score, negatively impacting
the precision and recall provided by the OLAC search service.
%% Shortcoming 2
Second, many language archives are not yet participating in OLAC.  For
example, the American Philosophical Assocation (Philadelphia) and the
National Anthropological Archives (Washington DC) are major national
centers having substantial collections of early language documentation
for native American languages.  The Oriental Institute (Chicago) and
the School of Oriental and African Studies (London) have vast
collections covering Asian and African languages.  Similarly, many
smaller archives are yet to participate in OLAC, e.g. the
Archive of the Indigenous Languages of Latin America (Austin), and the
Memorial University of Newfoundland Folklore and Language Archive (St
John's).  Many linguistics departments have archives holding language
documentation collected over many decades by their staff, e.g. UCLA,
UCSB, U Chicago, ...
%% Shortcoming 3
Third, many of the participating archives are more accurately described
as digitization projects.  Once project funding and institutional
backing cease, the materials may no longer be accessible.  These
initiatives are not yet following best practices in digital archiving,
as defined by the OAIS Reference Model \citep{OAIS02}.

%%%%%%%% Our vision for what needs to be done, long term...

In order to address these three shortcomings, we believe that the
language archives community needs to develop in three ways:

\begin{enumerate}
\item all OLAC repositories should have up-to-date catalogs
      that contain metadata conforming to best practice;
\item all major language archives should be in OLAC;
\item all OLAC archives should conform to current best practices
      in the treatment of their holdings.
\end{enumerate}

\subsection{Access to Language Documentation on the Web}

Automatic language identification
\citep{HughesBaldwin06lrec}

Finding language resources on the web
\citep{BaldwinBird06}

\vspace{1in}

\subsection{Access to Digital Content}

\vspace{1in}

