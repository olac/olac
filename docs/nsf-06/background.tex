\section{Background and Results from Prior Support}

Five years ago, OLAC, the Open Language Archives Community, was
founded with the goals of (i) developing consensus on best current
practice for the digital archiving of language resources, and (ii)
developing a network of interoperating repositories and services for
housing and accessing such resources.  For the first goal, substantial
progress on identifying best practices has now been achieved under the
NSF EMELD project; consideration of best practices is now incorporated
into the funding of language documentation projects (DEL,
\citet{BirdSimons03language}).  For the second goal, over 30 repositories
holding over 30,000 metadata records are participating in OLAC;
linguists can search all repositories simultaneously using OLAC
services, and the interest level now stands at 100,000 queries per
month.

Despite these early successes, it is still true today that ``new
digital language resources of all kinds -- lexicons, interlinear
texts, grammars, language maps, field notes, recordings -- are
difficult to reuse and less portable than the conventional printed
resources they replace'' \citep{BirdSimons03language}.  The proliferation of
language ``archiving projects'' have only made this problem more
acute: endangered languages are captured in ephemeral file formats,
and the so-called ``archive'' goes away at the end of the project.

Mirroring the twin goals of OLAC, there are just two things a linguist
must do to ensure that digital data endure: ``(1) the materials must be
put into an enduring file format, and (2) the materials must be
deposited with an archive that will make a practice of migrating them
to new storage media as needed'' (Simons 2006).  This amounts to an
injunction to individuals to *use* portable formats and credible
archives.  However, this change in practice will not happen unless
there are substantial, immediate benefits for the working linguist.

\vspace{1in}

\todo{Big picture: problem statement, vision for what needs to be done
  long-term to address the problem, approx 1 page; focus on access and
  the three different kinds of access reflected in our goals}

\vspace{1in}

\textbf{Prior Support:} The current project represents an outgrowth of
prior NSF support to the investigators in the following projects:
\textit{Multidimensional Exploration of Linguistic Databases}
  (Award \#9983258, 02/01/2000--01/31/2003);
\textit{TalkBank: A Multimedia Database of Communicative Interactions}
  (Award \#9978056, 10/01/1999--09/30/2005)
\textit{E-MELD: Electronic Metastructure for Endangered Languages Data}
  (Award \#0094934, 07/01/2001--06/30/2006);
\textit{The Rosetta Project- ALL Language Archive}
  (Award \#0333727, 11/01/2003--02/28/2006);
\textit{Querying Linguistic Databases}
  (Award \#0317826, 08/01/2003--07/31/2007).
Key results from these projects were:
(i) data models for linguistic annotations of text and speech
\citep{BirdLiberman01,MaedaBird00,GraffBird00,CottonBird02,CieriBird01,ATLAS00,BirdHarrington01}
(ii) open source linguistic annotation software
\citep{Bird01acl,MaedaBird02,BirdMaeda02,MaLee02};
(iii) OLAC, a community-wide resource discovery framework
\citep{BirdSimons00,BirdSimons00survey,BirdSimons01,BirdSimons02workshop,Simons02query,SimonsBird03lht,BirdSimons03chum,Simons03display,SimonsBird03llc,BirdSimons04metadata};
(iv) recommendations for digital storage of language data
\citep{BirdSimons03language};
(v) three CD-ROM publications of linguistic field data:
\citep{BirdBell01,Bird03paradigms,Bird03ngomba};
and
(vi) models and tools for querying linguistic databases
\citep{BirdBuneman01,BirdBunemanTan00,LaiBird04,Bird05planx,Bird06icde}

\todo{Add more citations for other PIs to the above}

\todo{Next sub-sections lay the groundwork for our three goals,
  probably 1 page on each}

\subsection{Access to Language Archive Catalogs}

\vspace{1in}

\subsection{Access to Language Documentation on the Web}

\vspace{1in}

\subsection{Access to Digital Content}

\vspace{1in}

