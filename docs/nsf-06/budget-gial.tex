\documentclass[11pt]{nsf}
\pagestyle{empty}
\setlength{\parskip}{.5ex}
\begin{document}

\begin{center}\textbf{\Large
Budget Justification (GIAL)
}\end{center}

Staff at the Graduate Institute of Applied Linguistics (GIAL) will
collaborate with University of Pennsylvania staff in pursuit of the
objectives laid out in the proposal narrative.
Staffing at GIAL will consist of a co-principal investigator (Simons)
and two graduate research assistants.  They will carry out the tasks
detailed below.

\vspace{1ex}

{\small\noindent
\begin{tabular}{lllp{3in}}
\textbf{Item} &
\textbf{Investigator} & \textbf{Allocation} & \textbf{Role} \\ \hline

A1 & Simons & 1.5 months/year &
Simons will direct the work of the two GRAs and collaborate with Bird
on the design of new technologies for deployment on the OLAC site.\\

B3 & Graduate Research Assistant 1 & half-time, years 1, 2, 3 &
The first GRA will be a student who is pursuing a Masters degree (in 
applied linguistics or language development) who is also
a competent Java programmer. The focus of activity will be
the software development tasks under Objective 3 that add
support for language resources and OLAC to the open-source DSpace
institutional repository system. The GRA will also participate
in writing and presenting one conference paper each year
to disseminate the results. \\

B3 & Graduate Research Assistant 2 & half-time, years 1, 2, 3 &
The second GRA will be a student who is pursuing a Masters degree (in 
applied linguistics or language development) and who has a good grasp of web technologies and
metadata concepts. The focus of activity will be
the tasks under Objective 1 that involve personal interaction
with actual and potential OLAC participants.  This will include
working with current OLAC participants to improve the quality of their
metadata (Outcome 1.1), identifying potential participants and advising
them through the process of developing a metadata repository and registering with
OLAC (Outcome 1.2), and working with participating archives to document
and improve their digital archiving parctices (Outcome 1.3). The GRA will also participate
in writing and presenting one conference paper each year
to disseminate the results.\\

E1 & Domestic travel & 6000, years 1, 2, 3 &
The Co-PI with a GRA will attend two conferences each year in digital archives or
language resources, to disseminate the results of our research as
widely as possible.\\

G1 & Materials and supplies & 2000, years 1 and 2 &
These funds are designated for equipment purchase. In the first year they will be used
to equip the GRAs with workstations.  In the second year they will
be used to purchase a server on which the language-resource adaptations to
DSpace can be mounted to set up an experimental institutional repository
for trial use by GIAL faculty and staff.\\

\end{tabular}}

\end{document}
