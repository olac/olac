\section{Methods and Workplan}

The discussion of methods and workplan is organized according to
each of the three objectives.  Each objective will be served by
three subtasks, each taking approximately one year to complete.

\todo{expand these items, add references, add any connecting logic}

\subsection*{Objective 1: Access to Language Resources in Archives}

\emph{Develop guidelines and services that encourage conformance to best
practices in cataloging and curating the world's language resources
in language archives.}

\def\task{1.1}
\paragraph{Task {\task} All OLAC repositories should have up-to-date catalogs
      that contain metadata conforming to best practice.}

\begin{enumerate}[label=\emph{\task\alph*}]
\item \emph{Best practice document:}
  adopt a recommendation document on metadata best practice
\item \emph{Report cards:}
  update the report cards to conform to the adopted recommendations
\item \emph{Metadata quality:}
  identify low-scoring archives, and work with them to
  improve the quality of their metadata, particularly the
  use of OLAC vocabularies; map vocabulary items; identify synonyms etc;
  develop an OLAC metadata usage note based on this experience
\item \emph{Updating archives:}
  identify stale archives, and work with them to bring their 
  catalog up-to-date
\item \emph{Quarterly reporting:}
  develop automated quarterly report to be sent to curators to 
  remind them about the quality and currency of their metadata, and 
  inform them of the top queries pertaining to their archives
\end{enumerate}

\def\task{1.2}
\paragraph{Task {\task} All major language archives should be participating in OLAC.}

\begin{enumerate}[label=\emph{\task\alph*}]
\item \emph{Non-participating archives:}
  compile a list of all known non-participating language archives
\item \emph{Repository editor:}
  configure XMLmind to create a static repository editor suitable
  for use by small archives
\item \emph{Collaboration:}
  contact all identified archives and consult to develop best
  strategy for generating their metadata
\item \emph{Scraping:}
  automatic scraping of HTML metadata, to show archivists
  their search results coming up; motivate them to improve precision
  and recall by mapping their terms to OLAC terms
\item \emph{OLAC export:}
  work with these archives to expose their catalogue in OLAC format
\end{enumerate}

\def\task{1.3}
\paragraph{Task {\task} All OLAC repositories should conform to current best practices
      for the long-term curation of their holdings.}

\begin{enumerate}[label=\emph{\task\alph*}]
\item \emph{Recommendation:}
  develop an OLAC recommendation document on this topic
\item \emph{Archive categories:}
  refine OLAC's collection level description to include
  more fine-grained categorization of OLAC; where this is vetted
  by OLAC against published criteria (derived from OAIS reference model)
\item \emph{Report card:}
  incorporate criteria into score card
\item \emph{Report card:}
  generalize the score card mechanism to facilitate the addition
  of new measures over time (as best practices evolve)
\end{enumerate}

\subsection*{Objective 2: Access to Language Resources on the Web}

\emph{Develop services to bridge the resource catalogs of the
  repository, library, and web domains (OAI, MARC, Google)
  to facilitate language resource discovery with precision through OLAC.}

\def\task{2.1}
\paragraph{Task {\task} Low density language materials identified in allied projects
      should be reliably categorized with OLAC vocabularies.}

\begin{enumerate}[label=\emph{\task\alph*}]
\item \emph{Crosswalk:}
  develop an OAI-DC to OLAC crosswalk
\item \emph{Metadata enrichment:}
  automatically identify language resources and
  enrich the metadata records (especially language id)
\end{enumerate}

\def\task{2.2}
\paragraph{Task {\task} All OAI and library holdings relevant to language
      documentation should be indexed in OLAC, by
      crosswalking and enriching OAI-DC and MARC records.}

\begin{enumerate}[label=\emph{\task\alph*}]
\item interface to a Z39.50 gateway (e.g. using JZKit,
   \url{http://developer.k-int.com/jzkit2/}
\item \emph{Crosswalk:}
  develop a MARC-OLAC crosswalk
\item \emph{Metadata enrichment:}
  automatically identify language resources and
  enrich the metadata records (especially language id)
\end{enumerate}

% JZKit:
% Note that the same Ian Ibbotson is the contact:
% http://www.loc.gov/z3950/agency/resources/software.html

% \item modify OLAC search to group results by item not by location

\def\task{2.3}
\paragraph{Task {\task} Web search engines should index all OLAC records,
      so that users who discover language resources using a web search
      quickly find OLAC records and are drawn to the OLAC site for
      more precise searching.}

\begin{enumerate}[label=\emph{\task\alph*}]
\item \emph{Static HTML:}
  generate static HTML pages for all OLAC metadata
\item \emph{Synonyms:}
  load static pages with synonyms for all identified linguistic
  terminology, so that these pages are more likely to appear in
  conventional web searches
\item \emph{Embedded queries:}
  enrich static pages with OLAC queries, so that users who discover
  OLAC via a Google search result are encouraged to remain on the
  OLAC site for more precise searching.
\end{enumerate}

\subsection*{Objective 3: Access to Language Resources from the Field}

\emph{Extend institutional repository software to support the best-practice
    cataloging and curation of language resources 
    uploaded by linguists themselves.}

\def\task{3.1}
\paragraph{Task {\task} Users of an institutional repository should be able
  to specify language identification with precision.}

\begin{enumerate}[label=\emph{\task\alph*}]
\item \emph{ISO 639-3 browser:}
  create an ISO 639-3 module for DSpace, including a flexible
  browser for 7000+ languages, with geographical, genetic, historical
  navigation, support ISO 639-3 as a vocabulary both for language and subject,
  add support for maintenance of the vocabulary over time
\item \emph{ISO 639-3 advocacy:}
  advocate ISO 639-3 to DC and DL communities (even general cataloging
  community?) for ISO 639-3 as a standard for language and subject
\item \emph{DSpace/ISO 639-3 sandbox:}
  set up a DSpace sandbox at the LDC, with user accounts,
  to permit the community to interact with the repository and give feedback
\item \emph{Dspace/ISO 639-3 production instance:}
  set up a production Dspace repository at GIAL that will serve
  as an example of an operational installation 
\end{enumerate}

\def\task{3.2}
\paragraph{Task {\task} Users of an institutional repository should be
    able to select an OLAC module so that they can integrate OLAC
    metadata creation into the workflow and lifecycle of their
    digital objects.}

\begin{enumerate}[label=\emph{\task\alph*}]
\item \emph{DSpace/OLAC module}
  create an OLAC module for DSpace, permitting users to create OLAC
  records and archive language resources
\item add support for the ingest of best practice formats
\item \emph{DSpace/OLAC sandbox:}
  extend the DSpace sandbox at the LDC with the OLAC module,
  to permit the community to interact with the repository and give feedback.
\item \emph{Dspace/OLAC production instance:}
  extend the production Dspace repository at GIAL with the OLAC
  module, to serve as an example of an operational installation 
\end{enumerate}

\def\task{3.3}
\paragraph{Task {\task} Field linguists should be able to participate in all the
    above while they are in the field.}

\begin{enumerate}[label=\emph{\task\alph*}]
\item \emph{Offline Ingest}
  add offline ingest via DSpace, extending the existing ``batch
  item importer,'' permitting a field linguist to
  prepare a submission information package (SIP) including OLAC
  metadata along with the full content, and upload it to an institutional repository
\item \emph{Distribution:}
  build and release a downloadable package, including XML editor,
  stylesheets, schemas, and configuration files, so that linguists
  can install the system on their personal machine
\end{enumerate}

\section{Dissemination Plans}
\label{sec:dissemination}

We will present the results of our research at major international
conferences in linguistics, computational linguistics, databases, and
digital libraries, and submit extended discussions of the research to
leading journals.  All tools and data created by the project will be
disseminated with open source and open content licenses.

