\section{Methods and Workplan}

The discussion of methods and workplan is organized according to
the three major objectives listed in the introduction.
Each objective is in turn elucidated in terms of three outcomes
(as listed at the conclusion of each of the preceding subsections).
The basic plan is that outcomes 1.1, 2.1, and 3.1 will be pursued
during year 1; outcomes 1.2, 2.2, and 3.2 during year 2;
and outcomes 1.3, 2.3, and 3.3 during year 3.
The following discussion fleshes out the nine outcomes in terms of the
tasks that must be performed in order to achieve them. The allocation of activities to project
personnel is discussed in the budget justification.


\subsection*{Objective 1: Access to Language Resources in Archives}

\emph{Develop guidelines and services that encourage conformance to best
practices in cataloging and curating the world's language resources
in language archives.}

\def\task{1.1}
\paragraph{Outcome {\task}: All OLAC repositories should have up-to-date catalogs
      that contain metadata conforming to best practice.}

\begin{enumerate}[label=\emph{\task\alph*}]\setlength{\itemsep}{0pt}
\item \emph{Best practice document:}
  Prepare and adopt an OLAC recommendation on metadata best practice,
  adapting the DC Usage Guidelines and fleshing out published recommendations
  \citep{BirdSimons03language,Simons06}.
\item \emph{Report cards:}
  Update automated score cards to conform to the adopted
  recommendations, by running a series of tests on metadata records.
\item \emph{Metadata quality:}
  Identify low-scoring archives and work with them to
  improve the quality of their metadata, particularly the
  use of OLAC vocabularies. 
  Develop an OLAC metadata usage note based on this experience.
\item \emph{Updating archives:}
  Identify OLAC archives whose metadata records have gone stale,
  and work with them to bring their catalogs up-to-date and to
  set up automatic processes to expose their live catalogs.
\item \emph{Quarterly reporting:}
  Develop an automated quarterly report to be emailed to curators to 
  remind them about the quality and currency of their metadata, and to
  inform them of statistics concerning queries on their archives.
\item \emph{Metrics:}
  Develop metrics for monitoring the size, coverage, and use of the 
  complete OLAC metadata catalog, and implement such reporting on the web site.

\end{enumerate}

\def\task{1.2}
\paragraph{Outcome {\task}: All major language archives should be participating in OLAC.}

\begin{enumerate}[label=\emph{\task\alph*}]\setlength{\itemsep}{0pt}
\item \emph{Non-participating archives:}
  Compile a list of all known non-participating language archives.
\item \emph{Collaboration:}
  Contact all identified archives and consult with them to develop best
  strategy for generating their metadata.
\item \emph{Scraping:}
  Do automatic scraping of the HTML metadata already published on
  archive sites and perform a quick low-accuracy conversion to OLAC
  metadata in order to show archivists their search results coming up
  in user queries; use this to show archivists the value of mapping their
  terms to OLAC terms and publishing OLAC metadata in order to
  improve precision and recall.
\item \emph{OLAC export:}
  Work with these archives to expose their catalog in OLAC format.
\item \emph{Repository editor:}
  Configure an open source XML editor (like XMLmind) to create a static repository editor
  suitable for use by small archives.
\end{enumerate}

\def\task{1.3}
\paragraph{Outcome {\task}: All OLAC repositories should conform to current best practices
      for the long-term curation of their holdings.}

\begin{enumerate}[label=\emph{\task\alph*}]\setlength{\itemsep}{0pt}
\item \emph{Recommendation:}
  Prepare and adopt an OLAC recommendation document on 
  best practices for the long-term curation of language archive holdings.
\item \emph{Archive categories:}
  Refine the {\tt <olac-archive>} description to include
  more fine-grained categorization of participating archives.
  Categories should be based on published criteria (in the OAIS reference model)
  and assignment of categories should be vetted by OLAC.
\item \emph{Report cards:}
  Incorporate these criteria into the automated score cards.
  Generalize the score card mechanism to facilitate the addition
  of new measures over time (as best practices evolve).
\end{enumerate}

\subsection*{Objective 2: Access to Language Resources on the Web}

\emph{Develop services to bridge the resource catalogs of the
  repository, library, and web domains (OAI, MARC, Google)
  to facilitate language resource discovery with precision through OLAC.}

\def\task{2.1}
\paragraph{Outcome {\task}:  All repository and library holdings relevant to language
   documentation should be indexed in OLAC, by
   crosswalking and enriching existing catalog records.}

\begin{enumerate}[label=\emph{\task\alph*}]\setlength{\itemsep}{0pt}
\item \emph{Language identification:}
  Develop an automated procedure for finding
  language identification in a catalog record and translating it to an
  appropriate ISO 639-3 code.
\item \emph{Data type identification:}
  Develop an automated procedure for finding
  linguistic data type information in a catalog record and translating
  it to an appropriate OLAC linguistic data type code.
\item \emph{OAI Crosswalk:}
  Develop an OLAC data provider that harvests from the
  leading OAI aggregator and serves up all records that are determined to
  be language resources with the metadata enriched for precise
  identification of language and linguistic data type (primary text vs
  lexicon vs linguistic description).
\item \emph{Z39.50 Crosswalk:}
  Develop an OLAC data provider that harvests from one or more
  Z39.50 gateways (e.g.~using JZKit), eliminates duplicates,
  applies the MARC to DC crosswalk\footnote{\scriptsize\url{http://www.loc.gov/marc/marc2dc.html}},
  and serves up all records that are determined to
  be language resources with the metadata enriched for precise
  identification of language and linguistic data type.
\end{enumerate}

% JZKit:
% Ian Ibbotson is the contact:
% http://www.loc.gov/z3950/agency/resources/software.html

\def\task{2.2}
\paragraph{Outcome {\task}: Low density language materials identified by
  linguistic web mining should be reliably categorized with OLAC vocabularies.}

\begin{enumerate}[label=\emph{\task\alph*}]\setlength{\itemsep}{0pt}
\item \emph{Language identification:}
  Develop an automated procedure for finding
  low-density language identification from an arbitrary web document,
  (e.g.~using character encodings, keywords,
  n-gram signatures, anchor text, and place-name references)
  and translating this to an appropriate ISO 639-3 code.
\item \emph{Data type identification:}
  Develop an automated procedure for identifying
  the linguistic data type of an arbitrary web document
  (e.g.~using Bayesian classifiers trained on existing metadata records)
  and translating this to an appropriate OLAC linguistic data type code.
\item \emph{Web mining:}
  Generate OLAC metadata records for low-density
  language resources discovered by linguistic web-mining projects,
  such as Hughes' collection of 600,000 low-density language URLs.
\end{enumerate}

% \item modify OLAC search to group results by item not by location

\def\task{2.3}
\paragraph{Outcome {\task}: Web search engines should index all OLAC records,
      so that users who discover language resources using a web search
      quickly find OLAC records and are drawn to the OLAC site for
      more precise searching.}

\begin{enumerate}[label=\emph{\task\alph*}]\setlength{\itemsep}{0pt}
\item \emph{Static HTML:}
  Generate a static HTML page for each OLAC metadata record, by adding
  a new module to the OLAC harvester.
\item \emph{Synonyms:}
  Add synonyms for all identified linguistic terminology to static HTML pages,
  so that these pages are more likely to be found in conventional web searches.
\item \emph{Embedded queries:}
  Enrich static pages with OLAC query links, so that users who discover
  OLAC records via a Google search result are encouraged to remain on the
  OLAC site for more precise searching.
\end{enumerate}

\subsection*{Objective 3: Access to Language Resources from the Field}

\emph{Extend institutional repository software to support the best-practice
    cataloging and curation of language resources 
    uploaded by linguists themselves.}

\def\task{3.1}
\paragraph{Outcome {\task}: Users of an institutional repository should be able
  to specify language identification with precision.}

\begin{enumerate}[label=\emph{\task\alph*}]\setlength{\itemsep}{0pt}
\item \emph{ISO 639-3 support:}
  Create an ISO 639-3 module for DSpace, including a flexible
  browser for 7000+ languages, with geographical, genetic, historical
  navigation. Support ISO 639-3 as a vocabulary both for language and subject.
  Build in support for maintaining the vocabulary over time.
\item \emph{ISO 639-3 advocacy:}
  Advocate ISO 639-3 to Dublin Core, digital library, and even general cataloging
  communities. Advocate ISO 639-3 as a standard for both language and subject.
\item \emph{DSpace/ISO 639-3 sandbox:}
  Set up a DSpace sandbox at the LDC, with guest accounts,
  to permit the community to interact with the
  ISO 639-3 enabled repository and give feedback.
\end{enumerate}

\def\task{3.2}
\paragraph{Outcome {\task}: Users of an institutional repository should be
    able to select an OLAC module so that they can integrate OLAC
    metadata creation into the workflow and lifecycle of their
    digital objects.}

\begin{enumerate}[label=\emph{\task\alph*}]\setlength{\itemsep}{0pt}
\item \emph{DSpace/OLAC module:}
  Create an OLAC module for DSpace, permitting users to create OLAC
  records for self-archiving of language resources.
\item \emph{More formats:}
  Add support for the ingest of best-practice formats for language
  documentation that are not yet supported.
\item \emph{DSpace/OLAC sandbox:}
  Extend the DSpace sandbox at the LDC with the OLAC module,
  to permit the community to interact with the repository and give feedback.
\item \emph{Dspace/OLAC production instance:}
  Set up a production Dspace repository at GIAL with the OLAC module, to serve
  as an example of an operational installation. 
\end{enumerate}

\def\task{3.3}
\paragraph{Outcome {\task}: Field linguists should be able to participate in all the
    above while they are in the field.}

\begin{enumerate}[label=\emph{\task\alph*}]\setlength{\itemsep}{0pt}
\item \emph{Offline ingest:}
  Develop a tool to permit a linguist in the field (while offline) to
  prepare a submission information package (SIP) including OLAC
  metadata and the full content. 
  Extend the existing ``batch item importer'' in DSpace to 
  upload such a SIP when an internet connection is available.
\item \emph{Distribution:}
  Build and release a downloadable package, including SIP editor,
  stylesheets, schemas, and configuration files, so that linguists
  can install the tool on their personal machines.
\end{enumerate}

\section{Dissemination Plans}
\label{sec:dissemination}

We will present the results of our research at major international
conferences in linguistics, computational linguistics, and
digital libraries, and submit extended discussions of the research to
leading journals.  The results of our research will also be deployed
on the OLAC web site as they come into existence. In this way the
results will be put into action immediately by the institutions
who are contributing their resources to the OLAC catalog and by 
individuals who are using the OLAC search services to find them.
All tools and data created by the project will be
disseminated with open source and open content licenses.
We will make occasional use of LanguageLog, when appropriate, to
disseminate information of broader interest that is discovered using
OLAC search services.
