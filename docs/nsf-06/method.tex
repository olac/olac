\section{Objectives}

\todo{These should just be copied over from the start of the background section.}

\subsection*{Objective 1: Access to Language Archive Catalogs}

\begin{quote}
{\it
  One sentence statement.
}
\end{quote}

\subsection*{Objective 2: Access to Language Documentation on the Web}

\begin{quote}
{\it
  One sentence statement.
}
\end{quote}

\subsection*{Objective 3: Access to Digital Content}

\begin{quote}
{\it
  One sentence statement.
}
\end{quote}

\section{Methods and Workplan}

The discussion of methods and workplan is organized according to
each of the three objectives.  Each objective will be served by
three subtasks, each taking approximately one year to complete.

\todo{expand these items, add references, add any connecting logic}

\subsection*{Objective 1: Access to Language Archive Catalogs}

\def\task{1.1}
\paragraph{Task {\task} All OLAC repositories should have up-to-date catalogs
      that contain metadata conforming to best practice.}

\begin{enumerate}[label=\textit{\task\alph*}]
\item \textit{Best practice document:}
  adopt a recommendation document on metadata best practice
\item \textit{Report cards:}
  update the report cards to conform to the adopted recommendations
\item \textit{Metadata quality:}
  identify low-scoring archives, and work with them to
  improve the quality of their metadata, particularly the
  use of OLAC vocabularies; map vocabulary items; identify synonyms etc;
  develop an OLAC metadata usage note based on this experience
\item \textit{Updating archives:}
  identify stale archives, and work with them to bring their 
  catalog up-to-date
\item \textit{Quarterly reporting:}
  develop automated quarterly report to be sent to curators to 
  remind them about the quality and currency of their metadata, and 
  inform them of the top queries pertaining to their archives.
\end{enumerate}

\def\task{1.2}
\paragraph{Task {\task} All major language archives should be participating in OLAC.}

\begin{enumerate}[label=\textit{\task\alph*}]
\item \textit{Non-participating archives:}
  compile a list of all known non-participating language archives
\item \textit{Repository editor:}
  configure XMLmind to create a static repository editor suitable
  for use by small archives
\item \textit{Collaboration:}
  contact all identified archives and consult to develop best
  strategy for generating their metadata
\item \textit{Scraping:}
  automatic scraping of HTML metadata, to show archivists
  their search results coming up; motivate them to improve precision
  and recall by mapping their terms to OLAC terms
\item \textit{OLAC export:}
  work with these archives to expose their catalogue in OLAC format
\end{enumerate}

\def\task{1.3}
\paragraph{Task {\task} All OLAC repositories should conform to current best practices
      in the treatment of their holdings.}

\begin{enumerate}[label=\textit{\task\alph*}]
\item \textit{Recommendation:}
  develop an OLAC recommendation document on this topic
\item \textit{Archive categories:}
  refine OLAC's collection level description to include
  more fine-grained categorization of OLAC; where this is vetted
  by OLAC against published criteria (derived from OAIS reference model)
\item \textit{Report card:}
  incorporate criteria into score card
\item \textit{Report card:}
  generalize the score card mechanism to facilitate the addition
  of new measures over time (as best practices evolve)
\end{enumerate}

\subsection*{Objective 2: Access to Language Documentation on the Web}

\def\task{2.1}
\paragraph{Task {\task} Low density language materials identified in allied projects
      should be reliably categorized for OLAC vocabularies.}

\begin{enumerate}[label=\textit{\task\alph*}]
\item OAI-DC to OLAC crosswalk
\item identify relevant records
\item automatic enrichment of records (especially language id)
\end{enumerate}

\def\task{2.2}
\paragraph{Task {\task} All OAI and library holdings relevant to language
      documentation should be indexed in OLAC, by
      crosswalking and enriching OAI-DC and MARC records.}

\begin{enumerate}[label=\textit{\task\alph*}]
\item deploy Z39.50 gateway software on OLAC site
\item build MARC-OLAC crosswalk
\item modify OLAC search to group results by item not by location
\end{enumerate}

\def\task{2.3}
\paragraph{Task {\task} Web search engines should index all OLAC records, so that
      users who discover language resources using a web search
      quickly find OLAC records and remain on the OLAC site for
      more precise searching.}

\begin{enumerate}[label=\textit{\task\alph*}]
\item generate static HTML pages for all OLAC metadata
  loaded with synonyms and containing embedded queries
\end{enumerate}

\subsection*{Objective 3: Access to Digital Content}

\def\task{3.1}
\paragraph{Task {\task} Users of popular institutional repository solutions should be able
  to specify language identification with precision.}

\begin{enumerate}[label=\textit{\task\alph*}]
\item create an ISO 639-3 module for DSpace, including a flexible
    browser for 7000+ languages, with geographical, genetic, historical
    navigation
\item support ISO 639-3 as a vocabulary both for language and subject
\item advocacy in DC and DL communities (even general cataloging
    community?) for ISO 639-3 as a standard for language and subject
\item support for maintenance of the vocabulary over time.
\item set up a DSpace sandbox at the LDC, with user accounts,
    to permit the community to interact with the repository and give feedback.
\item ? Set up a production Dspace repository at GIAL that will serve
    as an example of an operational installation 
\end{enumerate}

\def\task{3.2}
\paragraph{Task {\task} Users of popular institutional repository solutions should be
    able to install an OLAC module so that they can integrate OLAC
    metadata creation into the work-flows and life-cycles of their
    digital objects.}

\begin{enumerate}[label=\textit{\task\alph*}]
\item create an OLAC plugin???
\item add support for the ingest of best practice formats
\end{enumerate}

\def\task{3.3}
\paragraph{Task {\task} Field linguists should be able to participate in all the
    above while they are in the field.}

\begin{enumerate}[label=\textit{\task\alph*}]
\item add off-line ingest via DSpace, permitting a field linguist to
    prepare a SIP (including OLAC metadata) to be uploaded to an
    institutional repository.
\item build a downloadable package (including XML editor,
    stylesheets, schemas, and configuration files).
\end{enumerate}

Fortunately, this should be straightforward with DSpace since it
includes an application named ``batch item importer'' that ingests an
external Submission Information Package consisting of an XML metadata
document with the content files.

\section{Dissemination Plans}
\label{sec:dissemination}

We will present the results of our research at major international
conferences in linguistics, computational linguistics, databases, and
digital libraries, and submit extended discussions of the research to
leading journals.  All tools and data created by the project will be
disseminated with open source and open content licenses.

