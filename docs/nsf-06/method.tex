\section{Methods and Workplan}

The discussion of methods and workplan is organized according to
the three major objectives.  Each objective will be served by
three tasks, each taking approximately one year to complete.
The basic plan is that tasks 1.1, 2.1, and 3.1 will be pursued
during year 1; tasks 1.2, 2.2, and 3.2 during year 2; and tasks 1.3, 2.3, and 3.3 during year 3.
The following discussion fleshes out the nine tasks in terms of the
subtasks that have been identified. The allocation of activities to project
personnel is discussed in the budget justification.


\subsection*{Objective 1: Access to Language Resources in Archives}

\emph{Develop guidelines and services that encourage conformance to best
practices in cataloging and curating the world's language resources
in language archives.}

\def\task{1.1}
\paragraph{Task {\task}: All OLAC repositories should have up-to-date catalogs
      that contain metadata conforming to best practice.}

\begin{enumerate}[label=\emph{\task\alph*}]
\item \emph{Best practice document:}
  Prepare and adopt an OLAC recommendation on metadata best practice.
\item \emph{Report cards:}
  Update automated score cards to conform to the adopted recommendations.
\item \emph{Metadata quality:}
  Identify low-scoring archives and work with them to
  improve the quality of their metadata, particularly the
  use of OLAC vocabularies. 
  Develop an OLAC metadata usage note based on this experience.
\item \emph{Updating archives:}
  Identify stale archives, and work with them to bring their 
  catalogs up-to-date.
\item \emph{Quarterly reporting:}
  Develop an automated quarterly report to be emailed to curators to 
  remind them about the quality and currency of their metadata, and to
  inform them of statistics concerning queries on their archives.
\item \emph{Metrics:}
  Develop metrics for monitoring the size, coverage, and use of the 
  complete OLAC metadata catalog, and implement such reporting on the web site.

\end{enumerate}

\def\task{1.2}
\paragraph{Task {\task}: All major language archives should be participating in OLAC.}

\begin{enumerate}[label=\emph{\task\alph*}]
\item \emph{Non-participating archives:}
  Compile a list of all known non-participating language archives.
\item \emph{Collaboration:}
  Contact all identified archives and consult with them to develop best
  strategy for generating their metadata.
\item \emph{Scraping:}
  Do automatic scraping of HTML metadata to show archivists
  their search results coming up; use as a motivation to improve precision
  and recall by mapping their terms to OLAC terms.
\item \emph{OLAC export:}
  Work with these archives to expose their catalog in OLAC format.
\item \emph{Repository editor:}
  Configure an open source XML editor (like XMLmind) to create a static repository editor
  suitable for use by small archives.
\end{enumerate}

\def\task{1.3}
\paragraph{Task {\task}: All OLAC repositories should conform to current best practices
      for the long-term curation of their holdings.}

\begin{enumerate}[label=\emph{\task\alph*}]
\item \emph{Recommendation:}
  Prepare and adopt an OLAC recommendation document on this topic.
\item \emph{Archive categories:}
  Refine the {\tt <olac-archive>} description to include
  more fine-grained categorization of participating archives.
  Categories should be based on published criteria (in OAIS reference model)
  and assignment of categories should be vetted by OLAC.
\item \emph{Report cards:}
  Incorporate these criteria into the automated score cards.
  Generalize the score card mechanism to facilitate the addition
  of new measures over time (as best practices evolve).
\end{enumerate}

\subsection*{Objective 2: Access to Language Resources on the Web}

\emph{Develop services to bridge the resource catalogs of the
  repository, library, and web domains (OAI, MARC, Google)
  to facilitate language resource discovery with precision through OLAC.}

\def\task{2.1}
\paragraph{Task {\task}: Low density language materials identified in allied projects
      should be reliably categorized with OLAC vocabularies.}

\begin{enumerate}[label=\emph{\task\alph*}]
\item \emph{Crosswalk:}
  Develop an OAI-DC to OLAC crosswalk that adds precision for linguistic data
  type and language identification.
\item \emph{Metadata enrichment:}
  Automatically identify language resources and
  enrich the metadata records (especially language id).
\end{enumerate}

\def\task{2.2}
\paragraph{Task {\task}: All OAI and library holdings relevant to language
      documentation should be indexed in OLAC, by
      crosswalking and enriching OAI-DC and MARC records.}

\begin{enumerate}[label=\emph{\task\alph*}]
\item \emph{Gateway:}
  Interface to a Z39.50 gateway (e.g. using JZKit,
  \url{http://developer.k-int.com/jzkit2/}
\item \emph{Crosswalk:}
  Develop a MARC to OLAC crosswalk.
\item \emph{Metadata enrichment:}
  Automatically identify language resources and
  enrich the metadata records (especially language id).
\end{enumerate}

% JZKit:
% Note that the same Ian Ibbotson is the contact:
% http://www.loc.gov/z3950/agency/resources/software.html

% \item modify OLAC search to group results by item not by location

\def\task{2.3}
\paragraph{Task {\task}: Web search engines should index all OLAC records,
      so that users who discover language resources using a web search
      quickly find OLAC records and are drawn to the OLAC site for
      more precise searching.}

\begin{enumerate}[label=\emph{\task\alph*}]
\item \emph{Static HTML:}
  Generate static HTML pages for all OLAC metadata.
\item \emph{Synonyms:}
  Add synonyms for all identified linguistic terminology to static pages,
  so that these pages are more likely to appear in
  conventional web searches.
\item \emph{Embedded queries:}
  Enrich static pages with OLAC queries, so that users who discover
  OLAC via a Google search result are encouraged to remain on the
  OLAC site for more precise searching.
\end{enumerate}

\subsection*{Objective 3: Access to Language Resources from the Field}

\emph{Extend institutional repository software to support the best-practice
    cataloging and curation of language resources 
    uploaded by linguists themselves.}

\def\task{3.1}
\paragraph{Task {\task}: Users of an institutional repository should be able
  to specify language identification with precision.}

\begin{enumerate}[label=\emph{\task\alph*}]
\item \emph{ISO 639-3 support:}
  Create an ISO 639-3 module for DSpace, including a flexible
  browser for 7000+ languages, with geographical, genetic, historical
  navigation. Support ISO 639-3 as a vocabulary both for language and subject.
  Build in support for maintaining the vocabulary over time.
\item \emph{ISO 639-3 advocacy:}
  Advocate ISO 639-3 to Dublin Core, digital library, and even general cataloging
  communities. Advocate ISO 639-3 as a standard for both language and subject.
\item \emph{DSpace/ISO 639-3 sandbox:}
  Set up a DSpace sandbox at the LDC, with guest accounts,
  to permit the community to interact with the
  ISO 639-3 enabled repository and give feedback.
\end{enumerate}

\def\task{3.2}
\paragraph{Task {\task}: Users of an institutional repository should be
    able to select an OLAC module so that they can integrate OLAC
    metadata creation into the workflow and lifecycle of their
    digital objects.}

\begin{enumerate}[label=\emph{\task\alph*}]
\item \emph{DSpace/OLAC module:}
  Create an OLAC module for DSpace, permitting users to create OLAC
  records for self-archiving of language resources.
\item \emph{More formats:}
  Add support for the ingest of missing best practice formats.
\item \emph{DSpace/OLAC sandbox:}
  Extend the DSpace sandbox at the LDC with the OLAC module,
  to permit the community to interact with the repository and give feedback.
\item \emph{Dspace/OLAC production instance:}
  Set up a production Dspace repository at GIAL with the OLAC module, to serve
  as an example of an operational installation. 
\end{enumerate}

\def\task{3.3}
\paragraph{Task {\task}: Field linguists should be able to participate in all the
    above while they are in the field.}

\begin{enumerate}[label=\emph{\task\alph*}]
\item \emph{Offline ingest:}
  Develop a tool to permit a linguist in the field (while offline) to
  prepare a submission information package (SIP) including OLAC
  metadata and the full content. 
  Extend the existing ``batch item importer'' in DSpace to 
  upload such a SIP when an internet connection is available.
\item \emph{Distribution:}
  Build and release a downloadable package, including SIP editor,
  stylesheets, schemas, and configuration files, so that linguists
  can install the tool on their personal machine.
\end{enumerate}

\section{Dissemination Plans}
\label{sec:dissemination}

We will present the results of our research at major international
conferences in linguistics, computational linguistics, and
digital libraries, and submit extended discussions of the research to
leading journals.  The results of our research will also be deployed
on the OLAC web site as they come into existence. In this way the
results will be put into action immediately by the institutions
who are contributing their resources to the OLAC catalog and by 
individuals who are using the OLAC search services to find them.
All tools and data created by the project will be
disseminated with open source and open content licenses. 

