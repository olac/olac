\documentclass[11pt]{nsf}
\pagestyle{empty}
\setlength{\parskip}{.5ex}
\addtolength{\textheight}{5ex}
\begin{document}

\begin{center}\textbf{\Large
Project Summary\\[2ex]
    OLAC: Accessing the World's Language Resources
}\end{center}

\section*{Intellectual Merit}

%
% Define the terminology
%
Language resources are the bread and butter of
language documentation and linguistic investigation.
They include
the primary objects of study such as texts and recordings,
the outputs of research such as dictionaries and grammars,
and the enabling technologies such as software tools and interchange standards.
Increasingly, these resources are maintained and distributed in
digital form.
%
% Problem
%
Although language resources are expensive to create,
they are seldom archived appropriately,
and are often difficult or impossible for others to re-use.
Moreover, searching on the web for language resources in many languages
is a hit-and-miss affair for three reasons:
(i) resources are online but inadequately described,
(ii) resources are hidden behind form-based interfaces, or
(iii) resources are out of reach, only being kept with the linguist's
private files.
%
% What's been done about the problem, plus shortcomings
%
The Open Language Archives Community (OLAC) is addressing these
problems by providing a standard set of language resource descriptors
and a portal that permits users to query dozens of language archives
simultaneously using a single search.  However, the current coverage
of OLAC is only the tip of the iceberg.  New research in language
archiving is needed in order to tap the wealth of
new digital library services and web-mining technologies, and
to make the discovered language resources maximally accessible to linguists.

%
% Our goals
%
The aim of the proposed project is to dramatically improve the language
resource community's access to resources by achieving an 
order-of-magnitude increase in the coverage of the OLAC catalog
and in the use of OLAC search services.
The project seeks to do so through three main areas of activity:
%
\begin{description}\setlength{\itemsep}{0pt}\setlength{\parskip}{0pt}
  \item[Access to Language Resources in Archives:]
    Develop guidelines and services that encourage conformance to best
    practices in cataloging and curating the world's language resources
    in language archives.

  \item[Access to Language Resources on the Web:]
    Develop services to bridge the resource catalogs of the
    repository, library, and web domains,
    to facilitate language resource discovery with precision through OLAC.

  \item[Access to Language Resources from the Field:]
    Extend institutional repository software to support the
    best-practice cataloging and curation of language resources
    uploaded by linguists themselves.
\end{description}

\section*{Broader Impacts}

The proposed research should have a broad impact across the field of
linguistics by developing an infrastructure that can link linguists to
resources in and about the thousands of languages in the world. But the
impact will go well beyond the linguistics community. Access to these
language resources will assist technologists who are endeavoring to make
our modern computer-based tools work with every language, not just a select
few.  Within the society at large, educators and students and curious
citizens will have access to a wealth of materials that demonstrate the
full range of linguistic diversity in the word. Another audience for access
to language resources are the actual speakers of all the world's languages.
In the case of endangered languages, access to language resources is an
important asset in the process of language revitalization. As a final
impact of the research, it is hoped that the advocacy for the use of ISO
639-3 to precisely identify the 7,500 known human languages, past and
present, in the practice of library and archives cataloging (as opposed to
the current standard which recognizes fewer than 400 languages) will begin
to move the major storehouses of knowledge around the world to
appropriately deal with linguistic diversity.

\end{document}
