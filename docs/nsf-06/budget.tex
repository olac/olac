\documentclass[11pt]{nsf}
\pagestyle{empty}
\setlength{\parskip}{.5ex}
\begin{document}

\begin{center}\textbf{\Large
Budget Justification
}\end{center}

The ambitious program of research set out in this proposal will be
undertaken by a research team consisting of computational linguists,
research students, and software developers.  The time allocation for
each individual has been set at the minimum feasible level.

\vspace{1ex}

{\small\noindent
\begin{tabular}{lllp{3in}}
\textbf{Item} &
\textbf{Investigator} & \textbf{Allocation} & \textbf{Role} \\ \hline

A1 & Bird & 1 month/year &
tasks... \\

A2 & Liberman & 0.25 months/year &
project oversight,
guidance on ???\\

A3 & Simons & 1.5 months/year &
tasks... \\

B2 & Programmer & 3.00/year, years 1, 2, 3 &
software prototyping, user testing and archival format development \\

?? & Research Student 1 & &
tasks ... \\

?? & Research Student 2 & &
tasks ... \\

E1 & Domestic travel & 9000, years 1, 2, 3 &
conference travel by GIAL staff \\

E2 & Foreign travel & 3000, years 1, 2, 3 &
dissemination of results at an international conference \\

?? & Hardware & 4000, year 1 &
... \\

\end{tabular}}
\vspace{1ex}

\todo{Discussion of each item, stronger case for anything likely to be
  questioned.}

% The budget includes funds for a full-time experienced programmer for
% the first two years.  The sheer complexity of the modeling and data
% integration tasks demand a level of maturity and sophistication not
% found in new computer science graduates.  Solid experience with a
% range of programming languages and internet technologies will be
% required from day one.  The implementation work will be closely
% integrated with research on data models and query languages.
% Graphical user-interfaces will be developed, but only to the level
% necessary for supporting the primary research, and will use
% rapid-prototyping methods.  The use of such a research programmer is
% central to the success of the project.

\end{document}
