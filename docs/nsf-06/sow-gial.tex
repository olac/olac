\documentclass[11pt]{nsf}
\pagestyle{empty}
\setlength{\parskip}{1ex}
\begin{document}

\begin{center}\textbf{\Large
Statement of Work --- Graduate Institute of Applied Linguistics \\[2ex]
\itshape{OLAC: Accessing the World's Linguistic Resources}\\[2ex]
}\end{center}

Staff at the Graduate Institute of Applied Linguistics (GIAL) will
collaborate with UPenn staff in pursuit of the following three
objectives as laid out in the proposal narrative:

\begin{enumerate}
\item \textbf{Access to Language Resources in the Archive:}
  Develop guidelines and services that encourage conformance to best
  practices in documenting and curating the world's language resources.
\item \textbf{Access to Language Documentation on the Web:}
  Develop services to bridge the resource catalogues of the
  language, archive, library, and web domains (OLAC, OAI, MARC,
  Google), to facilitate language resource discovery on the web.
\item \textbf{Access to Language Resources from the Field:}
  Extend institutional repository software to support the
  documentation and curation of language resources by linguists,
  in conformance with best practice archiving standards.
\end{enumerate}

Staffing at GIAL will consist of a co-principal investigator (Simons)
and two graduate research assistance.  They will carry out the tasks
detailed below.  Lines are indexed to the subcontract budget.

\begin{description}
\item[A1 -- Simons -- 1.5 months/year:]
Simons will direct the work of the two GRAs and collaborate with Bird
on the design of new technologies for deployment on the OLAC site.

\item[B3 -- Graduate Research Assistant 1 -- half-time:]
The first GRA will be a student who is pursuing a Masters degree in 
applied linguistics who is also
a competent Java programmer. The focus of activity will be
the software development tasks under Objective 3 that add
support for language resources and OLAC to the open-source DSpace
institutional repository system. The GRA will also participate
in writing and presenting one conference paper each year
to disseminate the results.

\item[B3 -- Graduate Research Assistant 2 -- half-time:]
The second GRA will be a student who is pursuing a Masters degree in 
applied linguistics and who has a good grasp of web technologies and
metadata concepts. The focus of activity will be
the tasks under Objective 1 that involve personal interaction
with actual and potential OLAC participants.  This will include
working with current OLAC participants to improve the quality of their
metadata, identifying potential participants and advising
them through the process of developing a metadata repository and joining
OLAC, and working with participating archives to document
and improve digital archiving parctices. The GRA will also participate
in writing and presenting one conference paper each year
to disseminate the results.

\item[E1 -- Domestic travel -- \$6000/year:]
The Co-PI with a GRA will attend two conferences each year in digital archives or
language resources, to disseminate the results of our research as
widely as possible.

\item[G1 -- Equipment -- \$2000 (years 1,2):]
In the first year these funds will be used
to equip the GRAs with workstations.  In the second year they will
be used to purchase a server on which the experimental institutional repository
can be mounted for testing by the GIAL faculty and staff.
\end{description}

\end{document}
