\documentclass[11pt]{nsf}
\pagestyle{empty}
\setlength{\parskip}{.5ex}
\addtolength{\textheight}{5ex}
\begin{document}

\begin{center}\textbf{\Large Penn Facilities}\end{center}

\section*{Computer}

The Linguistic Data Consortium (LDC) maintains several systems
supporting the following functions:

\begin{itemize}
\item Compute Cluster: 8 Opteron blade server with 8 GB of RAM
\item Web Server: Xeon server with 2 GB of RAM
\item Database Servers:
        MySQL -- Itanium server with 2 GB RAM;
        MySQL -- Dual Pentium III server with 1 GB of RAM;
        Oracle -- Dual Sparc64 server with 1 GB of RAM.
\item File Servers: Multiple servers supporting 45 TB of storage:
        Dynamic File server -- 2 Dual Opterons with 2 GB of RAM;
        Static Data File servers -- 12 Dual Opterons with 2 GB of RAM.
\item Backup System:
        Tape robot supporting 60 TB of backup via 4 tape drives;
        Backup server capable of 4 tape/2 disk simultaneous backups.
\item Network:
        Public:
               5 switched 100 Mbps Ethernet ports;
               Optic fibre to the Internet;
               CIDR /23 network (equivalent to 2 class-C networks).
        Private: 192 fully switched Gigabit Ethernet ports.
\end{itemize}

In addition we have approximately seventy (70)
Annotation/Transcription workstations running various operating
systems such as Solaris, Windows, FreeBSD, and linux. Sixty of these
workstations are collected in four common work areas of varying size.

\section*{Office}

The Linguistic Data Consortium has its offices on the top floor of
3600 Market Street, in Philadelphia's University City Science
Center. The eighth floor suite, with over 11,000 usable square feet,
was configured specifically for LDC with 22 single, double and triple
offices, large and small conference rooms, a recording booth, a focus
group room and six laboratories.


\end{document}
