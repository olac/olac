\documentclass[11pt]{nsf}
\pagestyle{empty}
\setlength{\parskip}{.5ex}
\addtolength{\textheight}{5ex}
\begin{document}

\begin{center}\textbf{\Large
Project Summary\\[2ex]
    OLAC: Accessing the World's Language Resources
}\end{center}

\section*{Intellectual Merit}

%
% Define the terminology
%
Language resources are the bread and butter of
language documentation and linguistic investigation.
They include
the primary objects of study such as texts and recordings,
the outputs of research such as dictionaries and grammars,
and the enabling technologies such as software tools and interchange standards.
Increasingly, these resources are maintained and distributed in
digital form.
%
% Problem
%
Although language resources are beginning to abound on the web,
they are often difficult or impossible for interested parties to find and use.
Searching on the web for language resources in many languages
is a hit-and-miss affair for three reasons:
(i) resources are housed in archives that have never put their catalog online,
(ii) resources are exposed to online search engines but inadequately described
so that searches do not retrieve desired results with precision, or
(iii) resources are exposed online but are hidden behind form-based interfaces 
such that search engines cannot find them.
%
% What's been done about the problem, plus shortcomings
%
The Open Language Archives Community (OLAC) is addressing these
problems by providing a standard set of language resource descriptors
and a portal that permits users to query dozens of language archives
simultaneously using a single search.  However, the current coverage
of OLAC is only the tip of the iceberg.  New research 
is needed in order to tap the wealth of
new digital library services and web-mining technologies, and
to make the discovered language resources maximally accessible to linguists.

%
% Our goals
%
The aim of the proposed project is to greatly improve access to
language resources for linguists and the broader communities of
interest, by achieving an order-of-magnitude increase in the coverage
of the OLAC catalog and in the use of OLAC search services.  The
project seeks to do so through two main areas of activity:
%
\begin{description}\setlength{\itemsep}{0pt}\setlength{\parskip}{0pt}
  \item[Access to Language Resources in Archives:]
    Develop guidelines and services that encourage best common
    practices among language archives that will 
    facilitate language resource discovery with precision through OLAC.

  \item[Access to Language Resources on the Web:]
    Develop services to bridge the resource catalogs of the
    repository, library, and web domains,
    to facilitate language resource discovery with precision through OLAC.

\end{description}

\section*{Broader Impacts}

The proposed research should have a broad impact across the field of
linguistics by developing an online service that gives linguists
access to resources for the thousands of languages in the world. But
the impact will extend well beyond the linguistics community. Access
to these language resources will assist technologists who are
endeavoring to make information technologies work with every language,
not just a select few.  It will also permit educators, students and
members of society at large to access a wealth of materials that
demonstrate the full range of linguistic diversity in the world.  Yet
another audience for access to language resources are the actual
speakers of all the world's languages.  In the case of endangered
languages, access to language resources is a critical asset in the
process of language revitalization.  The project will also serve to
advocate the widespread use of ISO 639-3 codes to precisely identify
the 7,500 known human languages, past and present.  This will
encourage reform in the practice of library and archive cataloging,
which currently recognizes fewer than 400 languages, and will begin
the process of helping the major storehouses of knowledge around the
world to appropriately deal with linguistic diversity.

\end{document}
